\documentclass[tikz]{standalone}
\usepackage{cmap}
\usepackage[T2A]{fontenc}
\usepackage[utf8x]{inputenc}
\usepackage[english, russian]{babel}
\usepackage{tikz,physics}
\usepackage{amsmath,amssymb,cmap,pgfplots,pgfplotstable}
\usetikzlibrary{arrows,calc,intersections}
\pgfplotsset{compat=newest}
\usetikzlibrary
    {
        decorations.pathreplacing,
        decorations.pathmorphing,
        decorations.markings,
        patterns,
        calc,
        scopes,
        arrows,
        fadings,
        through,
        shapes.misc,
        arrows.meta,
        3d,
        quotes,
        angles,
        babel
    }

% Draw line annotation
% Input:
%   #1 Line offset (optional)
%   #2 Line angle
%   #3 Line length
%   #5 Line label
% Example:
%   \lineann[1]{30}{2}{$L_1$}
\newcommand{\lineann}[4][0.5]{%
    \begin{scope}[rotate=#2, blue,inner sep=2pt]
        \draw[dashed, blue!40] (0,0) -- +(0,#1)
            node [coordinate, near end] (a) {};
        \draw[dashed, blue!40] (#3,0) -- +(0,#1)
            node [coordinate, near end] (b) {};
        \draw[|<->|] (a) -- node[fill=white] {#4} (b);
    \end{scope}
}

\tikzset{
  % style to apply some styles to each segment of a path
  on each segment/.style={
    decorate,
    decoration={
      show path construction,
      moveto code={},
      lineto code={
        \path [#1]
        (\tikzinputsegmentfirst) -- (\tikzinputsegmentlast);
      },
      curveto code={
        \path [#1] (\tikzinputsegmentfirst)
        .. controls
        (\tikzinputsegmentsupporta) and (\tikzinputsegmentsupportb)
        ..
        (\tikzinputsegmentlast);
      },
      closepath code={
        \path [#1]
        (\tikzinputsegmentfirst) -- (\tikzinputsegmentlast);
      },
    },
  },
  % style to add an arrow in the middle of a path
  down arrow/.style={postaction={decorate,decoration={
        markings,
        mark=at position .5 with {\arrow[#1]{latex}}
      }}},
  % style to add an arrow in the middle of a path
  up arrow/.style={postaction={decorate,decoration={
        markings,
        mark=at position .55 with {\arrow[#1]{>}}
      }}},
  wave/.style={
        decorate,decoration={snake,post length=1.4mm,amplitude=0.5mm,
        segment length=3mm},thick},
    ground/.style={
        % The border decoration is a path replacing decorator. 
        % For the interface style we want to draw the original path.
        % The postaction option is therefore used to ensure that the
        % border decoration is drawn *after* the original path.
        postaction={draw,decorate, black!70,decoration={border,angle=-45,
                    amplitude=0.2cm,segment length=2mm}}},
interface/.style={
        pattern = north east lines,
        draw    = none,
        pattern color=gray!60,          
    },
    tangent/.style={
        decoration={
            markings,% switch on markings
            mark=
                at position #1
                with
                {
                    \coordinate (tangent point-\pgfkeysvalueof{/pgf/decoration/mark info/sequence number}) at (0pt,0pt);
                    \coordinate (tangent unit vector-\pgfkeysvalueof{/pgf/decoration/mark info/sequence number}) at (1,0pt);
                    \coordinate (tangent orthogonal unit vector-\pgfkeysvalueof{/pgf/decoration/mark info/sequence number}) at (0pt,1);
                }
        },
        postaction=decorate
    },
    use tangent/.style={
        shift=(tangent point-#1),
        x=(tangent unit vector-#1),
        y=(tangent orthogonal unit vector-#1)
    },
    use tangent/.default=1
}

\tikzstyle{spring}=[thick,decorate,decoration={zigzag,pre length=0.1cm,post
 length=0.1cm,segment length=6}]


 \usepackage[outline]{contour}
\contourlength{2pt}

\tikzset{
  pics/carc/.style args={#1:#2:#3}{
    code={
      \draw[pic actions] (#1:#3) arc(#1:#2:#3);
    }
  }
}
\usepackage{ifthen}
\usetikzlibrary{shapes.geometric}
\begin{document}
\begin{tikzpicture}[x=1em, y=1em]

\draw[->] (-20,0) -- (4,0) node[right]{$z$};
% \draw[->] (0,-5) -- (0,5) node[left]{$x$};

\begin{scope}[rotate=-20]
\begin{scope}[xshift=3em,yshift={0.5*cos(20)*.9em}]
	\draw[fill=black] (-1.5,-4) node[below,xshift=.5em] {} rectangle  ++ (0.5,8);
	\draw[fill=white, opacity=0.8] (-3,-4) node[below,xshift=.5em] {} rectangle  ++ (1,8);
	\draw[dashed] (-9,0) -- (-1,0);
\end{scope}


	% \draw[line width=2pt,->,>=latex] (2,1) node[above]{$\vec{E}_2$} -- ++(-3.5,0);
	\draw[line width=1pt,<-,>=latex] (-4,1)  -- node[above]{$\vec{E}_1$} ++(-3.5,0);
	\draw[line width=1pt,->,>=latex] (-3,-.25)   --  ++(-3.5,0) node[below]{$\vec{E}_2$}; 

% \begin{scope}[rotate=40,yshift=-.5em,xshift=-2em]
% 	\draw[line width=2pt,<-,>=latex] (-2,0.5)  -- node[above=.3em]{$\vec{E}_3 \equiv \vec{E}_\text{сиг}$} ++(-3.5,0);	
% 	\draw[line width=2pt,->,>=latex, dashed] (-.5,-0.5)  -- node[below=.3em]{$\vec{E}_4 \equiv \vec{E}_\text{обр}$} ++(-3.5,0);	
% \end{scope}
\end{scope}

% \draw[fill=white, opacity=0.8] (-13,-1.5) rectangle ++(5,3);
\draw (-11.5,0) node [trapezium, trapezium angle=-60, minimum width=5em, draw, fill=white, opacity=0.8, align=center] {Активная \\ среда};
% \path (-13,-1) -- node[align=center] {} ++(5,2);

\draw[line width=1pt,<-,>=latex] (-16,1)  -- node[above]{$\vec{E}_\text{сиг}$} ++(-3.5,0);
\draw[line width=2.5pt,->,>=latex] (-16,-1)  -- node[below]{$\vec{E}_\text{обр}$} ++(-3.5,0);





% \draw[fill=black] (0,0) circle (0.5pt) node[right,yshift=-.75em]{0} coordinate (o);
% 
% \draw ({180-25}:7) circle (2pt) node[left]{${E_2}$} coordinate (e2) -- (o);
% \draw ({180+25}:7) circle (2pt) node[left]{${E_1}$} coordinate (e1) -- (o);

% \draw[line width=2pt,xshift=-1cm,->,>=latex] ({180-25}:7) -- node[above] {$\vec{E}_1$} ({180-25}:4);
% \draw[line width=1pt,xshift=-1cm,->,>=latex] ({180+25}:7) -- node[below] {$\vec{E}_3$} ({180+25}:4);
% \draw[line width=1pt,xshift=-1cm,<-,>=latex] ({25}:-1) -- node[above=5pt] {$\vec{E}_4 \sim \vec{E}_3^*$} ({25}:2);

% \draw[line width=2pt,xshift=1cm, yshift=-1cm,<-,>=latex] ({-25}:-1) -- node[above=5pt] {$\vec{E}_2 \sim \vec{E}_1^*$} ({-25}:2);
% \draw[thick,->] ({180+25}:7) -- node[below] {$\vec{k}_1$} ({180+25}:4);
% \draw[interface] (-6,0) rectangle ++(11.9,-.5);
% \draw[-] (0,-5) coordinate (mz) -- ++(0,11) coordinate (z);
% \draw[->] (-6.2,0) node[left] {$0$} -- ++(12,0) coordinate (y) node[right]{};

% \draw (5,0) node[above=0.2em] {$\rho,\, c$};
% \draw (5,0) node[below=0.2em] {$\rho_1,\, c_1$};

% \draw[->] (-6,6) -- ++(0,-10)  node[below]{$z$};

% \draw[opacity=0.3] (0,0) coordinate (o) -- (120:6);
% \draw[opacity=0.3] (0,0) -- (60:6);
% \draw[opacity=0.3] (0,0) -- (295:6);
% % \draw[opacity=0.3] (0,0) -- (340:6);

% \draw[black,thick,latex-] (0,0) -- node[midway] {\contour{white}{$\vec{k}$}} (120:4.5) coordinate (ki);
% \draw[black,thick,-latex] (0,0) -- node[midway] {\contour{white}{$\vec{k}$}} (60:4.5) coordinate (kr);

% \draw[blue,thick,-latex] (0,0) -- node[midway] {\contour{white}{$\vec{k}_1$}} (295:3.65) coordinate (kt);

% \draw pic["$\theta /  2$", draw=black, -, angle eccentricity=1.4, angle radius=1.2cm]     {angle=z--o--e1};
% \draw pic["$$", draw=black, -, angle eccentricity=1.5, angle radius=1.1cm]     {angle=e2--o--z};

% \draw pic["$\theta_i$", draw=black, double, angle eccentricity=1.2, angle radius=1.4cm]     {angle=mz--o--kt};


\end{tikzpicture}
\end{document}